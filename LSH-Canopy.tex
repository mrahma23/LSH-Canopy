	
%% bare_conf.tex
%% V1.3
%% 2007/01/11
%% by Michael Shell
%% See:
%% http://www.michaelshell.org/
%% for current contact information.
%%
%% This is a skeleton file demonstrating the use of IEEEtran.cls
%% (requires IEEEtran.cls version 1.7 or later) with an IEEE conference paper.
%%
%% Support sites:
%% http://www.michaelshell.org/tex/ieeetran/
%% http://www.ctan.org/tex-archive/macros/latex/contrib/IEEEtran/
%% and
%% http://www.ieee.org/

%%*************************************************************************
%% Legal Notice:
%% This code is offered as-is without any warranty either expressed or
%% implied; without even the implied warranty of MERCHANTABILITY or
%% FITNESS FOR A PARTICULAR PURPOSE! 
%% User assumes all risk.
%% In no event shall IEEE or any contributor to this code be liable for
%% any damages or losses, including, but not limited to, incidental,
%% consequential, or any other damages, resulting from the use or misuse
%% of any information contained here.
%%
%% All comments are the opinions of their respective authors and are not
%% necessarily endorsed by the IEEE.
%%
%% This work is distributed under the LaTeX Project Public License (LPPL)
%% ( http://www.latex-project.org/ ) version 1.3, and may be freely used,
%% distributed and modified. A copy of the LPPL, version 1.3, is included
%% in the base LaTeX documentation of all distributions of LaTeX released
%% 2003/12/01 or later.
%% Retain all contribution notices and credits.
%% ** Modified files should be clearly indicated as such, including  **
%% ** renaming them and changing author support contact information. **
%%
%% File list of work: IEEEtran.cls, IEEEtran_HOWTO.pdf, bare_adv.tex,
%%                    bare_conf.tex, bare_jrnl.tex, bare_jrnl_compsoc.tex
%%*************************************************************************

% *** Authors should verify (and, if needed, correct) their LaTeX system  ***
% *** with the testflow diagnostic prior to trusting their LaTeX platform ***
% *** with production work. IEEE's font choices can trigger bugs that do  ***
% *** not appear when using other class files.                            ***
% The testflow support page is at:
% http://www.michaelshell.org/tex/testflow/



% Note that the a4paper option is mainly intended so that authors in
% countries using A4 can easily print to A4 and see how their papers will
% look in print - the typesetting of the document will not typically be
% affected with changes in paper size (but the bottom and side margins will).
% Use the testflow package mentioned above to verify correct handling of
% both paper sizes by the user's LaTeX system.
%
% Also note that the "draftcls" or "draftclsnofoot", not "draft", option
% should be used if it is desired that the figures are to be displayed in
% draft mode.
%
\documentclass[10pt, conference, compsocconf]{IEEEtran}
% Add the compsocconf option for Computer Society conferences.
%
% If IEEEtran.cls has not been installed into the LaTeX system files,
% manually specify the path to it like:
% \documentclass[conference]{../sty/IEEEtran}





% Some very useful LaTeX packages include:
% (uncomment the ones you want to load)


% *** MISC UTILITY PACKAGES ***
%
%\usepackage{ifpdf}
% Heiko Oberdiek's ifpdf.sty is very useful if you need conditional
% compilation based on whether the output is pdf or dvi.
% usage:
% \ifpdf
%   % pdf code
% \else
%   % dvi code
% \fi
% The latest version of ifpdf.sty can be obtained from:
% http://www.ctan.org/tex-archive/macros/latex/contrib/oberdiek/
% Also, note that IEEEtran.cls V1.7 and later provides a builtin
% \ifCLASSINFOpdf conditional that works the same way.
% When switching from latex to pdflatex and vice-versa, the compiler may
% have to be run twice to clear warning/error messages.






% *** CITATION PACKAGES ***
%
%\usepackage{cite}
% cite.sty was written by Donald Arseneau
% V1.6 and later of IEEEtran pre-defines the format of the cite.sty package
% \cite{} output to follow that of IEEE. Loading the cite package will
% result in citation numbers being automatically sorted and properly
% "compressed/ranged". e.g., [1], [9], [2], [7], [5], [6] without using
% cite.sty will become [1], [2], [5]--[7], [9] using cite.sty. cite.sty's
% \cite will automatically add leading space, if needed. Use cite.sty's
% noadjust option (cite.sty V3.8 and later) if you want to turn this off.
% cite.sty is already installed on most LaTeX systems. Be sure and use
% version 4.0 (2003-05-27) and later if using hyperref.sty. cite.sty does
% not currently provide for hyperlinked citations.
% The latest version can be obtained at:
% http://www.ctan.org/tex-archive/macros/latex/contrib/cite/
% The documentation is contained in the cite.sty file itself.



\usepackage{arydshln}

%\usepackage{pdfpages}


% *** GRAPHICS RELATED PACKAGES ***
%
\ifCLASSINFOpdf
  \usepackage[pdftex]{graphicx}
  % declare the path(s) where your graphic files are
  \graphicspath{{../pdf/}{../jpeg/}}
  % and their extensions so you won't have to specify these with
  % every instance of \includegraphics
  \DeclareGraphicsExtensions{.pdf,.jpeg,.png}
\else
  % or other class option (dvipsone, dvipdf, if not using dvips). graphicx
  % will default to the driver specified in the system graphics.cfg if no
  % driver is specified.
  \usepackage[dvips]{graphicx}
  % declare the path(s) where your graphic files are
  \graphicspath{{../eps/}}
  % and their extensions so you won't have to specify these with
  % every instance of \includegraphics
  \DeclareGraphicsExtensions{.eps}
\fi
% graphicx was written by David Carlisle and Sebastian Rahtz. It is
% required if you want graphics, photos, etc. graphicx.sty is already
% installed on most LaTeX systems. The latest version and documentation can
% be obtained at: 
% http://www.ctan.org/tex-archive/macros/latex/required/graphics/
% Another good source of documentation is "Using Imported Graphics in
% LaTeX2e" by Keith Reckdahl which can be found as epslatex.ps or
% epslatex.pdf at: http://www.ctan.org/tex-archive/info/
%
% latex, and pdflatex in dvi mode, support graphics in encapsulated
% postscript (.eps) format. pdflatex in pdf mode supports graphics
% in .pdf, .jpeg, .png and .mps (metapost) formats. Users should ensure
% that all non-photo figures use a vector format (.eps, .pdf, .mps) and
% not a bitmapped formats (.jpeg, .png). IEEE frowns on bitmapped formats
% which can result in "jaggedy"/blurry rendering of lines and letters as
% well as large increases in file sizes.
%
% You can find documentation about the pdfTeX application at:
% http://www.tug.org/applications/pdftex





% *** MATH PACKAGES ***
%
\usepackage[cmex10]{amsmath}
% A popular package from the American Mathematical Society that provides
% many useful and powerful commands for dealing with mathematics. If using
% it, be sure to load this package with the cmex10 option to ensure that
% only type 1 fonts will utilized at all point sizes. Without this option,
% it is possible that some math symbols, particularly those within
% footnotes, will be rendered in bitmap form which will result in a
% document that can not be IEEE Xplore compliant!
%
% Also, note that the amsmath package sets \interdisplaylinepenalty to 10000
% thus preventing page breaks from occurring within multiline equations. Use:
%\interdisplaylinepenalty=2500
% after loading amsmath to restore such page breaks as IEEEtran.cls normally
% does. amsmath.sty is already installed on most LaTeX systems. The latest
% version and documentation can be obtained at:
% http://www.ctan.org/tex-archive/macros/latex/required/amslatex/math/
\usepackage{romannum}


\usepackage{algorithm}
\usepackage[algo2e]{algorithm2e}
% *** SPECIALIZED LIST PACKAGES ***
%
%\usepackage{algorithmic}
% algorithmic.sty was written by Peter Williams and Rogerio Brito.
% This package provides an algorithmic environment fo describing algorithms.
% You can use the algorithmic environment in-text or within a figure
% environment to provide for a floating algorithm. Do NOT use the algorithm
% floating environment provided by algorithm.sty (by the same authors) or
% algorithm2e.sty (by Christophe Fiorio) as IEEE does not use dedicated
% algorithm float types and packages that provide these will not provide
% correct IEEE style captions. The latest version and documentation of
% algorithmic.sty can be obtained at:
% http://www.ctan.org/tex-archive/macros/latex/contrib/algorithms/
% There is also a support site at:
% http://algorithms.berlios.de/index.html
% Also of interest may be the (relatively newer and more customizable)
% algorithmicx.sty package by Szasz Janos:
% http://www.ctan.org/tex-archive/macros/latex/contrib/algorithmicx/




% *** ALIGNMENT PACKAGES ***
%
%\usepackage{array}
% Frank Mittelbach's and David Carlisle's array.sty patches and improves
% the standard LaTeX2e array and tabular environments to provide better
% appearance and additional user controls. As the default LaTeX2e table
% generation code is lacking to the point of almost being broken with
% respect to the quality of the end results, all users are strongly
% advised to use an enhanced (at the very least that provided by array.sty)
% set of table tools. array.sty is already installed on most systems. The
% latest version and documentation can be obtained at:
% http://www.ctan.org/tex-archive/macros/latex/required/tools/


%\usepackage{mdwmath}
%\usepackage{mdwtab}
% Also highly recommended is Mark Wooding's extremely powerful MDW tools,
% especially mdwmath.sty and mdwtab.sty which are used to format equations
% and tables, respectively. The MDWtools set is already installed on most
% LaTeX systems. The lastest version and documentation is available at:
% http://www.ctan.org/tex-archive/macros/latex/contrib/mdwtools/


% IEEEtran contains the IEEEeqnarray family of commands that can be used to
% generate multiline equations as well as matrices, tables, etc., of high
% quality.


%\usepackage{eqparbox}
% Also of notable interest is Scott Pakin's eqparbox package for creating
% (automatically sized) equal width boxes - aka "natural width parboxes".
% Available at:
% http://www.ctan.org/tex-archive/macros/latex/contrib/eqparbox/





% *** SUBFIGURE PACKAGES ***
%\usepackage[tight,footnotesize]{subfigure}
% subfigure.sty was written by Steven Douglas Cochran. This package makes it
% easy to put subfigures in your figures. e.g., "Figure 1a and 1b". For IEEE
% work, it is a good idea to load it with the tight package option to reduce
% the amount of white space around the subfigures. subfigure.sty is already
% installed on most LaTeX systems. The latest version and documentation can
% be obtained at:
% http://www.ctan.org/tex-archive/obsolete/macros/latex/contrib/subfigure/
% subfigure.sty has been superceeded by subfig.sty.



%\usepackage[caption=false]{caption}
\usepackage[caption=false,font=footnotesize]{subfig}
% subfig.sty, also written by Steven Douglas Cochran, is the modern
% replacement for subfigure.sty. However, subfig.sty requires and
% automatically loads Axel Sommerfeldt's caption.sty which will override
% IEEEtran.cls handling of captions and this will result in nonIEEE style
% figure/table captions. To prevent this problem, be sure and preload
% caption.sty with its "caption=false" package option. This is will preserve
% IEEEtran.cls handing of captions. Version 1.3 (2005/06/28) and later 
% (recommended due to many improvements over 1.2) of subfig.sty supports
% the caption=false option directly:
%\usepackage[caption=false,font=footnotesize]{subfig}
%
% The latest version and documentation can be obtained at:
% http://www.ctan.org/tex-archive/macros/latex/contrib/subfig/
% The latest version and documentation of caption.sty can be obtained at:
% http://www.ctan.org/tex-archive/macros/latex/contrib/caption/




% *** FLOAT PACKAGES ***
%
%\usepackage{fixltx2e}
% fixltx2e, the successor to the earlier fix2col.sty, was written by
% Frank Mittelbach and David Carlisle. This package corrects a few problems
% in the LaTeX2e kernel, the most notable of which is that in current
% LaTeX2e releases, the ordering of single and double column floats is not
% guaranteed to be preserved. Thus, an unpatched LaTeX2e can allow a
% single column figure to be placed prior to an earlier double column
% figure. The latest version and documentation can be found at:
% http://www.ctan.org/tex-archive/macros/latex/base/



%\usepackage{stfloats}
% stfloats.sty was written by Sigitas Tolusis. This package gives LaTeX2e
% the ability to do double column floats at the bottom of the page as well
% as the top. (e.g., "\begin{figure*}[!b]" is not normally possible in
% LaTeX2e). It also provides a command:
%\fnbelowfloat
% to enable the placement of footnotes below bottom floats (the standard
% LaTeX2e kernel puts them above bottom floats). This is an invasive package
% which rewrites many portions of the LaTeX2e float routines. It may not work
% with other packages that modify the LaTeX2e float routines. The latest
% version and documentation can be obtained at:
% http://www.ctan.org/tex-archive/macros/latex/contrib/sttools/
% Documentation is contained in the stfloats.sty comments as well as in the
% presfull.pdf file. Do not use the stfloats baselinefloat ability as IEEE
% does not allow \baselineskip to stretch. Authors submitting work to the
% IEEE should note that IEEE rarely uses double column equations and
% that authors should try to avoid such use. Do not be tempted to use the
% cuted.sty or midfloat.sty packages (also by Sigitas Tolusis) as IEEE does
% not format its papers in such ways.





% *** PDF, URL AND HYPERLINK PACKAGES ***
%
%\usepackage{url}
% url.sty was written by Donald Arseneau. It provides better support for
% handling and breaking URLs. url.sty is already installed on most LaTeX
% systems. The latest version can be obtained at:
% http://www.ctan.org/tex-archive/macros/latex/contrib/misc/
% Read the url.sty source comments for usage information. Basically,
% \url{my_url_here}.





% *** Do not adjust lengths that control margins, column widths, etc. ***
% *** Do not use packages that alter fonts (such as pslatex).         ***
% There should be no need to do such things with IEEEtran.cls V1.6 and later.
% (Unless specifically asked to do so by the journal or conference you plan
% to submit to, of course. )


% correct bad hyphenation here
\hyphenation{op-tical net-works semi-conduc-tor}
\usepackage{subfloat}
%\usepackage[labelfont=bf]{caption}
%\usepackage{subcaption}
\usepackage{amsmath}
\usepackage{amssymb}
\usepackage{multirow}
\usepackage[flushleft]{threeparttable}

\begin{document}

\title{Clustering Metagenome Sequences Using Canopies}

\author{\IEEEauthorblockN{Mohammad Arifur Rahman, Nathan LaPierre, Huzefa Rangwala and Daniel Barbara}
\IEEEauthorblockA{
Department of Computer Science\\
George Mason University\\
Fairfax, VA, United States\\
Email: mrahma23@gmu.edu, nlapier2@gmu.edu, rangwala@cs.gmu.edu and dbarbara@gmu.edu}
}
% make the title area
\maketitle

\begin{abstract}
	
The advancement in genome technologies has allowed for the 
collective sequencing of co-existing microbial communities (Metagenomics), omnipresent 
across various environments like the soil, ocean and human body.
%
Metagenomics has 
spurred the development of several 
bioinformatics approaches for analyzing 
the diversity, abundance, function and role of the 
different organisms 
within these communities. 
%
%As an example,  metagenomics allows for the analysis of the pathogenic role played by the microbial organisms within the human host and has implications for betterment of health and drug discovery. 


We present 
an approximate  but fast 
clustering algorithm
for analyzing 
large-scale metagenomic data. Our approach 
achieves efficiency 
by first partitioning the large number of sequence reads into 
groups (called canopies) using a fast locality sensitive hashing based 
distance function. This 
initial approximate assignment of sequence reads to 
canopies is refined by using 
state-of-the-art
sequence clustering algorithms. This
two-phase 
approach allows for the
use  of 
our developed algorithms as a pre-processing phase for computationally expensive 
clustering 
algorithms. Using the clusters as surrogates for 
Operational Taxonomic Units (OTUs) we 
can estimate  the biodiversity within 
a community sample.

The Canopy-based clustering algorithm is evaluated on synthetic and real 
world 16S and whole 
metagenome  benchmarks. We demonstrate the ability 
of our proposed approach to  
determine meaningful OTU assignments and observe
significant speedup with regards 
to run time when combined with three 
different state-of-the-art sequence clustering algorithms. 


\end{abstract}

\begin{IEEEkeywords}
Clustering, Canopy, Metagenome, 16S, Biodiversity 
\end{IEEEkeywords}

% For peer review papers, you can put extra information on the cover
% page as needed:
% \ifCLASSOPTIONpeerreview
% \begin{center} \bfseries EDICS Category: 3-BBND \end{center}
% \fi
%
% For peerreview papers, this IEEEtran command inserts a page break and
% creates the second title. It will be ignored for other modes.
\IEEEpeerreviewmaketitle

\section{Introduction}
\label{intro}

% Microorganism and it's Importance + what is metagenome
% HR Editing [Nice work Mohammad]

A large portion of the  earth's biomass comprises trillions of tiny  microorganisms with varying  biodiversity and plays a crucial role in ensuring the stability of environment. Communities of interacting microbes  exists in several ecosystems varying from the soil, ocean and human body \cite{MARHumanMicro}; and though, most of them are beneficial to the host some are known to cause unwanted conditions and can be linked to cause of diseases in the host \cite{MARTurnbaugh}. 

\emph{Metagenomics} is the sequencing of the collective DNA  of microbial organisms coexisting as communities. Analyzing metagenomes has provided an unprecedented opportunity to understand the diversity, role and function of these organisms within clinical and ecological environments \cite{MARHumanGut}\cite{MARMihaiPop}.

% What is read + 16S/18S + OTU + OTU-Clustering 


State-of-the-art genome sequencing technologies do not deliver the complete genome of an organism (millions in length), but large number of short contiguous subsequences called \emph{reads} of length 75 to 500 (based on technology) in random order. Sequencing communities of genomes exacerbates the problem since reads from different microbes are mixed together and upfront  the composition of a community in terms of abundance and identity of bacteria is unknown \cite{MARMetaChallenge}. Further, sequencing technologies produce datasets that range from Gigabytes (GB) to Terabytes (TB) with their own idiosyncrasies.  Alternatively, targeted metagenome sequencing  that involves sequencing of \emph{marker genes} has been popular for the characterization of these communities. 16S and 18S sequences are repetitive marker genes within microbial genomes but have variations in their sequences that allow them to be separated into different taxonomic groups \cite{MAR16S}. 


Over the years, several unsupervised clustering algorithms have been developed and used for the analysis of large-scale targeted and whole metagenome sequences (reviewed in Section \ref{sec:Literature}). The unsupervised grouping  of similar sequences from the 16S/18S genes is referred as \emph{binning} and allows for the estimation of biodiversity within a sample. Binning leads to assignment of similar sequences within groups called \emph{Operational Taxonomic Units (OTUs)} \cite{MAROTU}. Microbial OTUs are generally ecologically consistent across the hosts regardless of OTU clustering approaches \cite{MAROTUConsistant}. 

% What have we done and how it is better

In this paper we develop an approximate but efficient clustering algorithm based on Canopy Clustering \cite{MARCanopy} which can be used as a pre-clustering  step with any state-of-the-art sequence clustering algorithms. Specifically, our approach identifies canopies with a greedy procedure and a fast sequence distance metric based on locality sensitive hashing \cite{MARLshRef2}. Sequences within the canopies are considered as an initial partition (or grouping) of data which can be further refined by applying expensive and accurate clustering algorithms. Our developed approach treats each canopies independently allowing for easy parallelization.

We present experimental results on real and synthetic metagenome sequence benchmarks and show that use of Canopy Clustering (CC) in combination with UCLUST \cite{MARuclust}, SUMACLUST \cite{MARSumaclust} and SWARM \cite{MARSwarm2} leads to improved run time efficiency by maximum of 12.19, 21.09 and 18.61 times respectively with accurate clustering results. We also show the performance of our developed approach is scalable for large datasets with increased number of processors. The source code is freely available on Github\footnote{https://github.com/mrahma23/LSH-Canopy} for public use. 
          

\section{Literature Review}
\label{sec:Literature}

Over the years several sequence clustering methods have been developed and used widely for metagenome sequences. A comprehensive survey by Kopylova et. al. \cite{MARopenDeNovo} benchmarks various sequence clustering algorithms  including UCLUST \cite{MARuclust}, SWARM \cite{MARSwarm2}, SUMACLUST \cite{MARSumaclust} and MOTHUR \cite{MARMothur}

CD-HIT \cite{MARCDhit} is general purpose sequence clustering algorithm that follows an incremental, greedy approach. CD-HIT uses pairwise sequence alignment to find similar sequences. UCLUST \cite{MARuclust} is similar to CD-HIT but achieves a significant speedup over CD-HIT by using seeds (fixed length gapless subsequences) for performing pairwise sequence comparisons. MC-LSH \cite{MARMetaLSH} utilizes an efficient locality sensitive based hashing function to approximate the pairwise sequence similarity. MC-MinH \cite{MARMcMinH} uses min-wise \cite{MARMinWise} hashing along with the greedy clustering to group 16S and whole metagenome sequences. Mash \cite{MAROtherMinH}, uses MinHash locality sensitive hashing to reduce large sequences to a representative sketch and estimate pairwise distances. Other methods for clustering sequence reads include TOSS \cite{MARToss}, AbundanceBin \cite{MARAbundant} and CompostBin \cite{MARCompost}. All unique kmers are first grouped in TOSS and then clusters are merged based on kmer repetitions. In AbundanceBin, reads are modeled as a mixture of Poisson distributions. Expectation Maximization (EM) algorithm is used to infer model parameters for the final clustering. Principal component analysis is used within CompostBin to project the data into a lower dimensional space, followed with a graph partitioning approach. MOTHUR \cite{MARMothur} uses a pairwise distance matrix as input and performs hierarchical clustering. An expensive sequence alignment is used between all pairs of input sequences for computing distances.

SWARM \cite{MARSwarm2} uses exhaustive single-linkage clustering based on optimal sequence alignment. Sequences that are less than a certain distance from any other other sequence in the cluster are clustered together. SWARM attempts to reduce the impact of clustering parameters on the resulting OTUs by avoiding arbitrary global clustering thresholds and input sequence ordering dependence. At first SWARM builds an initial set of OTUs is constructed by iteratively agglomerating similar amplicons. Then amplicon abundance values are used to reveal OTUs’ internal structures and to break them into sub-OTUs.

SUMACLUST \cite{MARSumaclust} follows similar approach as UCLUST. Based on greedy strategy SUMACLUST incrementally constructs clusters by comparing an abundance-ordered list of input sequences against the representative set of already-chosen sequences. Initially this list is empty. UCLUST and recently developed, SWARM and SUMACLUST are considered to be the state-of-the-art metagenome sequence clustering methods by a benchmarking study \cite{MARopenDeNovo}. 

In this study we developed an approximate but highly efficient sequence clustering algorithm  using Canopies \cite{MARCanopy}. The developed approach can be used as a pre-processing step for other state-of-art sequence clustering algorithms to improve efficiency and also retains correctness in clustering.

\section{Methods}
\label{sec:Methods}
\subsection{\textbf{Overview}}

Figure \ref{fig:flowchart} shows an overview of our proposed canopy clustering approach for metagenome sequences. Sequence reads are represented with \textit{kmers} defined as contiguous subsequence of length k. We select a read randomly as the canopy centroid. Then we compute the distances of other reads to this canopy centroid. Two thresholds are used in Canopy clustering algorithm. If the a distance is less than a \emph{tight} threshold ($T2$) we assign the read to the canopy and that read will not be considered for another canopy. Otherwise, if a distance is less than a \emph{soft} threshold ($T1$) we assign the read to the canopy but that read will be available for another canopy. This process continues until all sequence reads is assigned to at least one canopy. For fast canopy assignment we use a random projection based technique. Specifically, we use Locality Sensitive Hash (LSH) to compute pairwise distances. More accurate clustering methods are used to sub-cluster each canopy in parallel. Sub-clustering method considers only the members within the canopy which reduces pairwise distance calculations.

\begin{figure}
	\centering
	\includegraphics[width=\linewidth,height=10cm]{flowchart.pdf}	
	\caption{Workflow of Canopy Clustering for Large Scale Metagenome Data}
	\label{fig:flowchart}
\end{figure}

\subsection{\textbf{Canopy Clustering}}

Canopy Clustering \cite{MARCanopy} is an efficient approximate clustering algorithm often used as pre-processing step for other accurate and expensive clustering methods like the K-means algorithm or the Hierarchical clustering algorithm. It is intended to speed up clustering operations on large data sets, where standard clustering algorithms may be impractical in terms of run time and memory consumption due to the large number of pairwise distance calculations. For a dataset with $N$ instances, worst case calculations without canopy clustering is $N^2$. After canopy clustering, if the number of canopies is $k$ then worst case calculations with canopy clustering is $\sum_{i=1}^{k}(c_i)^2$ where $c_i$ is the number of instances within $i$th canopy.

Canopy clustering uses two distance thresholds, (\romannum{1}) \textit{soft} threshold $T1$ and (\romannum{2}) \textit{tight} threshold $T2$. If data point $p_1$ is within the soft distance threshold $T1$ with centroid $p_2$ then $p_1$ will reside in same canopy as $p_2$ but $p_1$ may belong to other canopies assuming that it has only met soft threshold and best match is yet to be found. Thus one data point may belong to multiple canopies in Canopy clustering. On the other hand if data point $p_1$ is within the tight distance threshold $T2$ with centroid $p_2$ then canopy clustering assigns $p_1$ to the same canopy as $p_2$ and stops assigning $p_1$ to any other canopy assuming that tight threshold has been met and best canopy assignment for $p_1$ has been found. Canopy centroids are selected randomly until all data points is assigned to at least one canopy. 


\subsection{\textbf{Locality Sensitive Hashing}}

Canopies are intended to reduce pairwise distance calculations. Canopy clustering itself should be efficient which requires a fast and approximate distance measure for Canopy assignments. Locality Sensitive Hashing (LSH) \cite{MARLshRef2} is an algorithm for solving the approximate or exact near neighbors search. Suppose we have a space S of points with a distance measure $d(x,y)$. A family H of hash functions is said to be $(d1, d2, p1, p2)$-sensitive for any x and y in S: if $d(x, y)<d1$, then the probability over all $h\in{H}$, that $h(x)=h(y)$ is at least $p1$ and if $d(x, y)>d2$ , then the probability over all $h\in{H}$, that $h(x)=h(y)$ is at most $p2$. LSH provides a fast approximate distance measure while reducing data dimensionality which makes LSH appropriate for Canopy clustering in 16S and whole metagenomic data.

We constructed the LSH family with bit sampling \cite{indyk1998approximate}. The normalized kmers frequency based feature vectors of sequence reads were first projected into $d$ dimensional vectors in $\{0,1\}^d$ space. Given an input vector $v$ and a random hyperplane defined by $r$, we let $h(v)=\{0,1\}$ based on $sgn(v\cdot{r})=\pm{1}$ that indicates on which side of the hyperplane $v$ lies. Each possible choice of $r$ defines a single function. Let $H$ be the set of all such functions. For any $h_i\in{H}$ and for any two data $x,y$ the probability that $x$ and $y$ \emph{agree} on $i^{th}$ positions of their respective $d$-length binary vector is
\begin{equation}
P[h_i(x)=h_i(y)]=1-\frac{distance(h_i(x),h_i(y))}{d} 
\end{equation}
Where \emph{distance} is the hamming distance and $d$ is the number of bits. Hence $H=\{h_1,h_2,...,h_d\}$ is a $(d_1,d_2,1-d_1/d,1-d2/d)$ sensitive LSH family. Both the random projection and hamming distance calculation is cheap and fast which makes this scheme appropriate for our purpose which is fast partitioning of large scale genome sequences so that more expensive clustering can be done in parallel.

%%Moreover, this projection and distance scheme is computationally inexpensive than another popular alternative for LSH known as MinHash \cite{broder1997resemblance}. MinHash provides $n!$ choices for hash functions in LSH family where $n$ is the total number of kgrams. Whereas our scheme generates only $d<<n!$ hash functions where $d$ can be easily estimated by sampling and validation based on 16S and metagenomic data.  

\subsection{\textbf{Sub-Clustering Inside Canopies}}
\label{sub-cluster}
Canopy clustering  makes initial approximate partitions of the dataset and reduces pairwise distance computations. Each of these partitions can be further sub clustered in parallel with more accurate and expensive clustering methods with lower computational cost since only the points inside canopies are taken into consideration. We have used three recent and popular sequence clustering methods as more accurate and expensive sub-clustering measure inside canopies in this study. UCLUST \cite{MARuclust}, SUMACLUST \cite{MARSumaclust} and SWARM \cite{MARSwarm2} were used for sub-clustering canopies.

\subsection{\textbf{Merging results from Canopies}}

Each cluster (OTU) is represented by the Longest Common Subsequence of all member sequences. The final step of our proposed framework is to merge OTU representations generated by canopies. According to Canopy cluster algorithm a single data point may belong to multiple canopies as long as the soft threshold is met. As a result similar OTU representations may appear from multiple canopies. To eliminate redundancy we run UCLUST on the OTU representations.
 
In the following Sections UCLUST, SUMACLUST and SWARM with Canopy clustering approach are represented as CC$_{UCLUST}$, CC$_{SUMACLUST}$ and CC$_{SWARM}$ respectively where the term CC stands for Canopy Clustering. 

\section{Experimental Evaluation}
\label{sec:Experimental}

\subsection{\textbf{Dataset Description}}

To evaluate the performance of our proposed approach we use previously published synthetic and real world 16S, 18S and metagenome sequence benchmarks. Key statistics and relevant information regarding these datasets are presented in Table \ref{table:finaltabledataset}. 

\begin{table}[htb] 
	%\centering 
	\caption{\textbf{Dataset Statistics}}
	\label{table:finaltabledataset}
	\resizebox{\columnwidth}{!}{%
	\begin{tabular}{|l| c c c c|} 
		\hline
		\multirow{2}{*}{{\bf{Datasets}}} & \multirow{2}{*}{{\bf{Type}}} & {\bf{\# of}} & {\bf{\# of}}  & \multirow{2}{*}{\bf{Platform}}\\
		 & & \bf{Reads} & \bf{Samples} & \\
		\hline
		{Bokulich$_2$} \cite{MARmockDatasetRef} & M & 6,938,836 & 4 & H\\
		{Bokulich$_3$} \cite{MARmockDatasetRef} & M & 3,594,237 & 4 & H\\
		{Bokulich$_6$} \cite{MARmockDatasetRef} & M & 250,903 & 1 & H\\
		{Canadian Soil} \cite{MARcanadianSoil} & R & 2,966,053 & 13 & H\\
		{Body Sites} \cite{MARbodySites} & R & 886,630 & 602 & G\\
		{Global Soil} \cite{MARglobalSoil} & R & 9,252,764 & 57 & H\\
		{Liver Cirrhosis} \cite{qin2014alterations} & R & 6,117,828,130 & 232 & H\\ 
		\hline
	\end{tabular}
	}
	\begin{tablenotes}
		\item Table shows information about dataset used in this study. M, R, H and G represent Mock, Real World, HiSeq and GS-FLX, respectively.		
	\end{tablenotes}
\end{table}	


\begin{table*}[t]
	\centering 
	\caption{\textbf{Performance Comparison [$F$-Score and Pearson Correlation Coefficient ($\rho$)]}} 
	\label{table:performanceTable}
	
	\scalebox{0.82}{
		
		\begin{tabular}{|l|c c c c| c c c c c|}
			
			\hline
			
			\textbf{Methods} & \textbf{Comparison Metric} & \multicolumn{8}{c|}{\textbf{Datasets}}\\
			
			\cline{3-10}
			
			& & \multicolumn{3}{c|}{\textit{Synthetic}} & \multicolumn{5}{c|}{\textit{Real World}}\\ 
			
			\cline{3-10}
			
			%&  & Bokulich$_2$ & Bokulich$_3$ & Bokulich$_6$ & Body Sites & Canadian Soil & Global Soil\\
			
			&  &  &  & & & & & Liver Cirrhosis & Liver Cirrhosis\\
			
			& & Bokulich$_2$ & Bokulich$_3$ & Bokulich$_6$ & Body Sites & Canadian Soil & Global Soil & Metagenome & Metagenome\\
			
			& & & & & & & & (Sampled 3M) & (Sampled 30M)\\
			
			\hline
			
			\multirow{1}{*}{UCLUST} & $F$-Measure & 0.39 & 0.40 & 0.51 & $N/A$ & $N/A$ & $N/A$ & $N/A$ & $N/A$\\
			
			\hdashline
			
			\multirow{2}{*}{CC$_{UCLUST}$} & $F$-Measure & \textit{\textbf{0.39}} & \textit{\textbf{0.41}} & \textit{\textbf{0.52}} & $N/A$ & $N/A$ & $N/A$ & $N/A$ & $N/A$\\
			& ($\rho$) with Respect to UCLUST & \textit{\textbf{0.9831}} & \textit{\textbf{0.9753}} & \textit{\textbf{0.9831}} & \textit{\textbf{0.9682}} & \textit{\textbf{0.8419}} & \textit{\textbf{0.9824}} & \textit{\textbf{0.9216}} & \textit{\textbf{0.9072}}\\
			
			\hline 
			
			\multirow{1}{*}{SUMACLUST} & $F$-Measure & 0.40 & 0.41 & 0.51 & $N/A$ & $N/A$ & $N/A$ & $N/A$ & $N/A$\\
			
			\hdashline
			
			\multirow{2}{*}{CC$_{SUMACLUST}$} & $F$-Measure & \textit{\textbf{0.41}} & \textit{\textbf{0.42}} & \textit{\textbf{0.51}} & $N/A$ & $N/A$ & $N/A$ & $N/A$ & $N/A$\\
			& ($\rho$) with Respect to SUMACLUST & \textit{\textbf{0.9709}} & \textit{\textbf{0.9813}} & \textit{\textbf{0.9538}} & \textit{\textbf{0.9518}} & \textit{\textbf{0.7643}} & \textit{\textbf{0.8714}} & \textit{\textbf{0.9281}} & \textit{\textbf{0.8614}}\\ 
			
			\hline
			
			\multirow{1}{*}{SWARM} & $F$-Measure & 0.46 & 0.48 & 0.55 & $N/A$ & $N/A$ & $N/A$ & $N/A$ & $N/A$\\
			
			\hdashline
			
			\multirow{2}{*}{CC$_{SWARM}$} & $F$-Measure & \textit{\textbf{0.46}} & \textit{\textbf{0.49}} & \textit{\textbf{0.56}} & $N/A$ & $N/A$ & $N/A$ & $N/A$ & $N/A$\\
			& ($\rho$) with Respect to SWARM & \textit{\textbf{0.9817}} & \textit{\textbf{0.97861}} & \textit{\textbf{0.9251}} & \textit{\textbf{0.9648}} & \textit{\textbf{0.7581}} & \textit{\textbf{0.9143}} & \textit{\textbf{0.9263}} & \textit{\textbf{0.8533}}\\
			
			\hline
			
		\end{tabular}
	}
	\small
	\begin{tablenotes}
		\item Table shows values of $F$-measures and Pearson Correlation Coefficient ($\rho$-value) of UCLUST, SUMACLUST, SWARM and their respective versions with Canopy clustering. $F$-measures is only available for synthetic datasets but not for real world datasets since no ground truths like known taxonomic profiles are available for them. $\rho$-value was calculated based on the taxonomy profiles at Genus level generated from clustered OTUs provided by a method and it's corresponding Canopy counterpart. Higher $F$-measures reflect better clustering by adhering to ground truth. Higher $\rho$-values reflect stronger correlation between taxonomic profiles. Higher F-scores and $\rho$ values are represented with bold and italic letters.         
	\end{tablenotes}
	
\end{table*}


\begin{table*}[t]
	\centering 
	\caption{\textbf{Biodiversity Comparison [Faith’s phylogenetic diversity metric (PD), Shannon and Simpson]}} 
	\label{table:bioDiversityTable}
	
	\scalebox{0.80}{
		
		\begin{tabular}{|c|c c c c| c c c c|}
			
			\hline
			
			\textbf{Methods} & \textbf{Comparison Metric} & \multicolumn{7}{c|}{\textbf{Datasets}}\\
			
			\cline{3-9}
			
			& & \multicolumn{3}{c|}{\textit{Synthetic}} & \multicolumn{4}{c|}{\textit{Real World}}\\ 
			
			\cline{3-9}
			
			&  &  &  & & & & & Liver Cirrhosis\\
			
			& & Bokulich$_2$ & Bokulich$_3$ & Bokulich$_6$ & Body Sites & Canadian Soil & Global Soil & Metagenome\\
			
			& & & & & & & & (Sampled 3M)\\
			
			\hline
			
			\multirow{3}{*}{UCLUST} & PD Range & $\left[171.95-221.85\right]$ & $\left[186.90-212.84\right]$ & $\left[104.51-104.51\right]$ & $\left[1.46-46.79\right]$ & $\left[0.30-1352.73\right]$ & $\left[2.98-3.29\right]$ & $\left[3.14-51.47\right]$\\ 
			& Shannon Range & $\left[2.52-3.51\right]$ & $\left[2.43-3.54\right]$ & $\left[5.87-5.87\right]$ & $\left[0.29-7.67\right]$ & $\left[2.32-7.86\right]$ & $\left[1.84-8.30\right]$ & $\left[0.27-5.77\right]$\\
			& Simpson Range & $\left[0.55-0.75\right]$ & $\left[0.55-0.76\right]$ & $\left[0.96-0.96\right]$ & $\left[0.049-0.98\right]$ & $\left[0.80-0.99\right]$ & $\left[0.0-0.98\right]$ & $\left[0.02-0.97\right]$\\
			
			& & & & & & & & \\
			
			\multirow{3}{*}{CC$_{UCLUST}$} & PD Range & $\left[164.79-217.72\right]$ & $\left[169.41-198.36\right]$ & $\left[109.33-109.33\right]$ & $\left[2.37-47.13\right]$ & $\left[0.52-1419.31\right]$ & $\left[3.02-3.81\right]$ & $\left[4.61-53.29\right]$\\
			& Shannon Range & $\left[2.61-3.83\right]$ & $\left[2.92-3.91\right]$ & $\left[6.41-6.41\right]$ & $\left[0.93-7.13\right]$ & $\left[3.26-7.61\right]$ & $\left[3.72-7.38\right]$ & $\left[0.13-6.17\right]$\\
			& Simpson Range & $\left[0.64-0.87\right]$ & $\left[0.56-0.93\right]$ & $\left[0.96-0.96\right]$ & $\left[0.081-0.99\right]$ & $\left[0.84-0.99\right]$ & $\left[0.21-0.99\right]$ & $\left[0.09-0.98\right]$\\
			
			\hline
			
			\multirow{3}{*}{SUMACLUST} & PD Range & $\left[106.00-162.78\right]$ & $\left[142.85-174.19\right]$ & $\left[89.22-89.22\right]$ & $\left[0.93-39.47\right]$ & $\left[0.59-1279.29\right]$ & $\left[2.98-3.29\right]$ & $\left[0.28-49.61\right]$\\
			& Shannon Range & $\left[2.00-3.01\right]$ & $\left[2.19-3.28\right]$ & $\left[5.48-5.48\right]$ & $\left[0.16-7.43\right]$ & $\left[2.32-7.32\right]$ & $\left[1.00-7.89\right]$ & $\left[1.37-7.83\right]$\\
			& Simpson Range & $\left[0.52-0.73\right]$ & $\left[0.54-0.75\right]$ & $\left[0.95-0.95\right]$ & $\left[0.027-0.98\right]$ & $\left[0.80-0.99\right]$ & $\left[0.40-0.98\right]$ & $\left[0.19-0.91\right]$\\
			
			& & & & & & & & \\
			
			\multirow{3}{*}{CC$_{SUMACLUST}$} & PD Range & $\left[114.96-171.57\right]$ & $\left[147.85-187.91\right]$ & $\left[93.81-93.81\right]$ & $\left[0.86-41.63\right]$ & $\left[0.81-1292.34\right]$ & $\left[1.37-4.89\right]$ & $\left[0.18-51.35\right]$\\
			& Shannon Range & $\left[2.51-3.94\right]$ & $\left[2.96-4.11\right]$ & $\left[5.94-5.94\right]$ & $\left[1.21-7.13\right]$ & $\left[3.12-7.79\right]$ & $\left[2.17-7.25\right]$ & $\left[1.91-7.39\right]$\\
			& Simpson Range & $\left[0.68-0.79\right]$ & $\left[0.51-0.74\right]$ & $\left[0.96-0.96\right]$ & $\left[0.06-0.99\right]$ & $\left[0.88-0.99\right]$ & $\left[0.23-0.98\right]$ & $\left[0.27-0.88\right]$\\
			
			\hline
			
			\multirow{3}{*}{SWARM} & PD Range & $\left[18.37-24.73\right]$ & $\left[17.36-19.81\right]$ & $\left[30.84-30.84\right]$ & $\left[1.44-28.66\right]$ & $\left[0.54-706.57\right]$ & $\left[5.79-6.18\right]$ & $\left[2.06-39.71\right]$\\ 
			& Shannon Range & $\left[2.98-3.91\right]$ & $\left[2.01-3.04\right]$ & $\left[5.03-5.03\right]$ & $\left[0.28-7.63\right]$ & $\left[1.0-7.79\right]$ & $\left[1.66-7.81\right]$ & $\left[1.47-6.91\right]$\\
			& Simpson Range & $\left[0.70-0.82\right]$ & $\left[0.53-0.74\right]$ & $\left[0.95-0.95\right]$ & $\left[0.05-0.98\right]$ & $\left[0.50-0.99\right]$ & $\left[0.00-0.99\right]$ & $\left[0.18-0.89\right]$\\
			
			& & & & & & & & \\
			
			\multirow{3}{*}{CC$_{SWARM}$} & PD Range & $\left[19.18-26.87\right]$ & $\left[18.43-22.61\right]$ & $\left[31.48-31.48\right]$ & $\left[2.34-29.97\right]$ & $\left[1.37-748.71\right]$ & $\left[2.34-8.46\right]$ & $\left[ 3.18-40.73\right]$\\
			& Shannon Range & $\left[1.66-4.87\right]$ & $\left[1.19-4.13\right]$ & $\left[6.03-6.03\right]$ & $\left[0.89-7.13\right]$ & $\left[2.81-8.06\right]$ & $\left[2.81-7.87\right]$ & $\left[2.03-6.88\right]$\\
			& Simpson Range & $\left[0.66-0.88\right]$ & $\left[0.41-0.81\right]$ & $\left[0.91-0.91\right]$ & $\left[0.11-0.99\right]$ & $\left[0.74-0.99\right]$ & $\left[0.14-0.99\right]$ & $\left[0.07-0.86\right]$\\
			
			\hline
			
		\end{tabular}
	}
	\small
	\begin{tablenotes}
		\item Table shows ranges of values for Faith’s Phylogeny Diversity (PD), Shannon and Simpson coefficient over all samples in a dataset in the format $[minimum-maximum]$. Most of these datasets contain multiple samples and Alpha diversity metrics like PD, Shannon and Simpson values are generated for each of these samples separately. Biodiversity metric values changes significantly over samples e.g. diversity from hair samples and teeth cavity are supposedly different. So instead of mean values this Table represents $[minimum-maximum]$  ranges of values a sample can take. Similar ranges reflect similar diversity.      
	\end{tablenotes}
	
\end{table*}

\begin{table*}
	\centering
	\caption{\textbf{Runtime Comparison (in minutes)}} 
	\label{table:Runtime}
	\scalebox{0.75}{
		\begin{tabular}{|c c c|c c c|c c c|c c c|} 
			
			\hline
			
			\multicolumn{3}{|c|}{\textbf{Datasets}} & \multicolumn{9}{c|}{\textbf{Methods}}\\
			%\cline{3-9}
			
			\hline
			
			\multirow{1}{*}{Type} & \multirow{1}{*}{Title} & \multirow{1}{*}{\# of Reads} & \multirow{1}{*}{UCLUST} & \multirow{1}{*}{CC$_{UCLUST}$} & \multirow{1}{*}{Speed Up} & \multirow{1}{*}{SUMACLUST} & \multirow{1}{*}{CC$_{SUMACLUST}$} & \multirow{1}{*}{Speed Up} & \multirow{1}{*}{SWARM} & \multirow{1}{*}{CC$_{SWARM}$} & \multirow{1}{*}{Speed Up}\\
			
			\hline
			
			\multirow{3}{*}{\textit{Synthetic}} & Bokulich$_2$ & 6,938,836 & 12.71 & \textit{\textbf{3.08}} & \textit{\textbf{4.13x}} & 114.53 & \textit{\textbf{13.89}} & \textit{\textbf{8.24x}} & 128.12 & \textit{\textbf{17.24}} & \textit{\textbf{7.43x}}\\ 
			      
			& Bokulich$_3$ & 3,594,237 & 10.43 & \textit{\textbf{3.61}} & \textit{\textbf{2.89x}} & 27.73 & \textit{\textbf{5.11}} & \textit{\textbf{5.43x}} & 18.37 & \textit{\textbf{3.39}} & \textit{\textbf{5.41x}}\\ 
			      
			& Bokulich$_6$ & 250,903 & 4.47 &  \textit{\textbf{2.09}} & \textit{\textbf{2.14x}} & 5.61 & \textit{\textbf{2.07}} & \textit{\textbf{2.71x}} & 4.17 & \textit{\textbf{1.29}} & \textit{\textbf{3.21x}}\\ 
			
			\hline
			
			\multirow{3}{*}{\textit{Real World}} & Body Sites & 886,630 & 9.46 & \textit{\textbf{3.03}} & \textit{\textbf{3.12x}} & 18.42 & \textit{\textbf{6.02}} & \textit{\textbf{3.06x}} & 16.37  & \textit{\textbf{5.83}} & \textit{\textbf{2.81x}}\\
			
			& Canadian Soil & 2,966,053 & 13.65 & \textit{\textbf{4.17}} & \textit{\textbf{3.27x}} & 363.96 & \textit{\textbf{56.61}} & \textit{\textbf{6.43x}} & 117.53 & \textit{\textbf{20.84}} & \textit{\textbf{5.64x}}\\
			
			& Global Soil & 9,252,764 & 108.21 &  \textit{\textbf{18.75}} & \textit{\textbf{5.77x}} & 510.92 & \textit{\textbf{45.37}} & \textit{\textbf{11.26x}} & 289.51 &  \textit{\textbf{34.84}} & \textit{\textbf{8.31x}}\\
			
			& Liver Cirrhosis Metagenome & 3,000,000 & 14.57 &  \textit{\textbf{4.03}} & \textit{\textbf{3.61x}} & 46.62 & \textit{\textbf{9.02}} & \textit{\textbf{5.17x}} & 41.37 &  \textit{\textbf{8.75}} & \textit{\textbf{4.73x}}\\
			
			& Liver Cirrhosis Metagenome & 30,000,000 & 22.41h &  \textit{\textbf{1.84h}} & \textit{\textbf{12.19x}} & 46.62h & \textit{\textbf{2.21h}} & \textit{\textbf{21.09x}} & 37.43h &  \textit{\textbf{2.01h}} & \textit{\textbf{18.61x}}\\		
			
			\hline 
			
		\end{tabular}
	}
\end{table*} 

\textbf{Synthetic Datasets:}

\subsubsection{\textit{Bokulich$_2$}}
This dataset was prepared using the 
Illumina TruSeq v2 paired--end library
preparation kit. It is a simulated 16S rRNA gene 
microbial community dataset.
This 
dataset contains 19 taxonomic Families, 19 Genera, 22 Species 
and 22 Strains in total. This dataset can also be 
found in the QIIME database (identifier 1685).

\subsubsection{\textit{Bokulich$_3$}}
Similar to Bokulich$_2$ except that it was 
prepared with the 
TruSeq v1 paired-end library kit at 
Illumina Cambridge and is  also available in the 
QIIME database (identifier 1686).

\subsubsection{\textit{Bokulich$_6$}}
This  16S rRNA dataset 
was sequenced at Washington University School of Medicine and contains evenly distributed microbial communities. This dataset contains 13 taxonomic Families, 23 Genera, 44 Species and 48 Strains in total.

All these datasets from Bokulich et al. \cite{MARmockDatasetRef} are available at QIIME database\footnote{http://qiime.org/home\_static/dataFiles.html} under their respective ID's. Since, these are simulated datasets the taxonomic profile of microbial organisms within them are known.

\textbf{Real World Datasets:}

\subsubsection{Canadian Soil}
The Canadian Soil dataset\footnote{http://www.cm2bl.org/} contains genomic data of soil spanning from Arctic Tundra to Agricultural soil suitable for different agricultural products.  

\subsubsection{Body Sites}
This dataset contains composition of bacterial communities from up to 27 different body sites in healthy adults. A collection of 602 samples acquired from different body sites of human subjects are provided with meta-data.

\subsubsection{Global Soil}
The global soil data was taken from Ramirez et al. \cite{MARglobalSoil} which is a study of the below-ground diversity in New York City's Central Park.

\subsubsection{Liver Cirrhosis}
The Liver Cirrhosis dataset was taken from the study \cite{qin2014alterations} which is a whole gut microbiome wide association study of stool samples from 98 liver cirrhosis patients and 83 healthy controls to characterize the fecal microbial communities and their functional composition. In total, 860 Gb of high-quality sequence data was generated in this study. Because of the high volume of sequence reads in this dataset, we used random sampling for feasible clustering performance evaluations. Three million and thirty million sequence reads were randomly chosen from original dataset regardless of sample labels (disease or control) for our study.   


\subsection{\textbf{Evaluation Metrics}}

We evaluate the performance of our developed clustering approach using 
the following commonly used metrics that are used 
for the assessment of (i) outputs from clustering algorithms, (ii) biodiversity
within metagenome samples and (iii) computational run time. 


\subsubsection{\textit{Faith’s phylogenetic diversity metric (PD)}}
Faith’s phylogenetic diversity \cite{MARfaith1992conservation} combines all 
the branch lengths of phylogenetic tree as a measure of diversity. So, if a new 
OTU is found and it is closely related to another OTU in the sample, it will contribute to 
a small increase to the PD score. However, if a new OTU from different lineage is found then it will contribute to a large increase in the PD score.

\subsubsection{\textit{Shannon Entropy}}

Shannon-Wiener diversity index is defined as:

\begin{equation}
H={-} \sum_{i=1}^{s} \left( p_i\log_2p_i \right)
\end{equation}

where s is the number of OTUs and $p_i$ is the proportion of the community represented by OTU $i$. The Shannon index increases as both the richness and evenness of the community increase. The fact that the index incorporates both components of biodiversity can be seen as both a strength and a weakness. It is a strength because it provides a simple summary, but it is a weakness because it makes it difficult to compare communities that differ greatly in richness.


\subsubsection{\textit{Simpson's Index}}
Simpson’s index is defined as ${1-dominance}$ or

\begin{equation}
1 - \sum p_i^2
\end{equation}

where where $p_i$ is the proportion of the community represented by OTU $i$. Simpsonʼs index is based on the probability that any two individuals drawn at random from an infinitely large community belong to the same species. It measures \textit{evenness} of the community from 0 to 1. Higher value of this index implies higher similarity and relatively lower diversity of microorganisms within a sample.


\subsubsection{\textit{F-measure}}
In case of synthetic datasets, expected taxonomic composition is known (\emph{ground truth}). False-positive (FP) refers to the number of taxonomy that was found in observed but not expected, false-negative (FN) refers to the number of taxonomy that exists in expected but not observed, and true-positive (TP) refers to the number of taxonomy exists in both observed and expected. The following definitions were used:

\begin{equation}
precision = \frac{TP}{(TP + FP)}
\end{equation}

\begin{equation}
recall = \frac{TP}{(TP + FN)}
\end{equation}

\begin{equation}
F Score = \frac{2 \times precision \times recall}{(precision + recall)}
\end{equation}

\subsubsection{\textit{Pearson Coefficient Correlation ($\rho$-value)}}
After getting Operational Taxonomic Units (OTU) from a clustering method we create a taxonomic profile at the Genus level. Pearson’s correlation coefficient was computed to measure the relatedness of taxonomic assignment between a pair of tools. Values range between -1 and 1, with -1 indicating a negative correlation, 0 indicating no correlation, and 1 indicating a positive correlation or strong relationship.

\subsection{\textbf{Parameter Settings}}
For kmers, the value of parameter k was set to 4. The parameters for bit length of LSH ($d$), canopy's soft ($T1$) and hard ($T2$) thresholds were set by using random sampling and validation. For validation purpose, only ten percent of data was randomly sampled from the dataset of interest. First the parameter for LSH bit length $d$ was estimated. In order to know correctness of our estimation of $d$, we performed Canopy clustering with corresponding sub clustering inside canopies and compared the results corresponding to the bit lengths. For synthetic dataset $F1$-scores and for real world datasets Pearson Coefficient Correlation ($\rho$) with results from corresponding expensive naive sequence clustering. We started with bit length $d$=1 and continued to increase value of $d$ as long as we got better results for sampled data. Initial value of soft ($T1$) and tight ($T2$) threshold  was set to 0.6 and 0.4 respectively where maximum distance is 1.0. These are relaxed initial values for both $T1$ and $T2$, meaning that they allow for higher repetitions of sequence reads in multiple canopies due to higher value of $T1$ and easy final canopy assignment due to higher value of $T2$ which results in rough and fast approximate clusters prior to expensive sub-clustering step. Any bit length $d$ that brings comparatively better results for these relaxed $T1$ and $T2$ on sampled data is expected to bring better results for more strict $T1$ and $T2$ on whole datasets.

Once $d$ is selected we estimated values for $T1$ and $T2$ on the same sampled data. First, hard threshold ($T2$) was estimated with soft threshold ($T1$) fixed at 0.6. $T2$ determines which sequence reads will be retained for next iteration of canopy clustering. Same comparison metrics from the estimation of parameter $d$ were used. We decreased the value of $T2$ from 0.4 (\emph{relaxed}) to 0.1 (\emph{strict}) with a step size of 0.01. The $T2$ corresponding to best result was selected. Then $T2$ was fixed at the best estimation and $T1$ was varied from 0.6 to best estimation for $T2$ with step size of 0.01. The value for $T1$ corresponding to best result was selected. Table \ref{table:parameters} shows the estimated parameter values for our study.   

\begin{table}[htb] 
	%\centering 
	\caption{\textbf{Parameter Settings}}
	\label{table:parameters}
	\resizebox{\columnwidth}{!}{%
		\begin{tabular}{|l| c c c c c|} 
			\hline
			Datasets & Total \# of & \# of Reads in &  & Parameters & \\
			\cline{4-6}			
			 & Reads & Sampled Data & $d$ & $T1$ & $T2$ \\
			\hline
			{Bokulich$_2$} & 6,938,836 & 693,884 & 47 & 0.48 & 0.36\\
			{Bokulich$_3$} & 3,594,237 & 359,428  & 31 & 0.43 & 0.34\\
			{Bokulich$_6$} & 250,903 & 25,090 & 17 & 0.37 & 0.21\\
			{Canadian Soil} & 2,966,053 & 296,605 & 29 & 0.46 & 0.37\\
			{Body Sites} & 886,630 & 88,663 & 22 & 0.39 & 0.22\\
			{Global Soil} & 9,252,764 & 925,276 & 59 & 0.49 & 0.38\\
			{Liver Cirrhosis} & 3,000,000 & 300,000 & 48 & 0.42 & 0.35\\
			{Liver Cirrhosis} & 30,000,000 & 3,000,000 & 67 & 0.51 & 0.39\\ 
			\hline
		\end{tabular}
	}
\end{table}	     

\subsection{\textbf{Hardware and Software for Experiments}}
We performed all the experiments on computers with Intel 5th generation Core i7 2.70GHz 64bit processor with 8 core CPUs and 12GB memory. Similarity identity of sub-clustering methods were set to default 97\%. For implementation we used Python 2.7.12 and QIIME \cite{MARQiime} version 1.9.0. Taxonomy for reported OTUs was assigned using the RDP Classifier \cite{MARRdp} against the 97\% representative databases for Greengenes \cite{MARGreen1} and Silva \cite{MARSilva} for methods used in this study. We used PyNast\footnote{http://biocore.github.io/pynast/} open source sequence aligner for aligning clustered output. 

\section{Results Discussion} 
\label{sec:Results}

\subsection{\textbf{Clustering Performance}}
Table \ref{table:performanceTable} shows the performance of  UCLUST, SUMACLUST and SWARM with their corresponding versions with CC clustering. Table \ref{table:performanceTable} compares F-scores and Pearson Correlation Coefficient ($\rho$). F-scores are only available for synthetic benchmarks since taxonomic profile for them is known as ground truth. Correlation values were generated based on taxonomic profiles at Genus level generated from outputs from clustering methods. From Table \ref{table:performanceTable} we can see that F-scores obtained from a clustering method and its corresponding Canopy clustering version are very similar and in some cases better. We see a higher F-score for Bokulich$_2$ benchmark from CC$_{SUMACLUST}$. CC$_{UCLUST}$ and CC$_{SWARM}$ provided same F-scores as their respective naive versions for Bokulich$_2$. For Bokulich$_3$ benchmark we observed higher F-scores for all Canopy clustering methods. Finally for Bokulich$_6$ benchmark F-scores of CC$_{UCLUST}$ and CC$_{SWARM}$ were improved comparing to respective naive versions.

For all benchmarks we observe strong correlations between taxonomic profiles at genus level. The highest correlation was observed for Bokulich$_6$ benchmark between UCLUST and CC$_{UCLUST}$ with 0.9831. From observed F-scores and correlation values we can say that running a sequence clustering method with our proposed LSH based Canopy cluster framework can bring similar or better results.


\subsection{\textbf{Biodiversity Estimation}} Clustering metagenome sequences outputs OTUs that represent biodiversity contained in the samples from which data is collected. We compare the biodiversity represented by OTUs from naive methods and their respective Canopy clustering variations used in this study. Table \ref{table:bioDiversityTable} shows Faith’s phylogenetic diversity metric (PD), Shannon and Simpson index after clustering with different methods. These metrics are some of the popular Alpha Diversity metrics meaning that they measure species diversity in sites or habitats at a local scale. These metric values are generated per sample basis. Table \ref{table:bioDiversityTable} shows ranges of metric values in $[minimum-maximum]$ format provided by methods with and without Canopy clustering. Any sample of a dataset will take value from this range. These ranges may not be same since OTUs vary over clustering methods. But they should be similar for a method and it's Canopy counterpart. We can see from Table \ref{table:bioDiversityTable} that Canopy clustering based methods produce similar ranges of values as their naive counterparts. No significant changes in diversity metric values were observed which indicates that Canopy based approaches reproduce similar biodiversity while reducing run time.

\subsection{\textbf{Runtime Comparison}} Table \ref{table:Runtime} shows the run time in minutes of UCLUST, SUMACLUST, SWARM and their respective versions with our proposed Canopy clustering pipeline. We see from Table \ref{table:Runtime} that CC$_{UCLUST}$ outperforms UCLUST for all datasets. The highest speed up for UCLUST was observed for the metagenome data containing 30 million randomly sampled sequence reads where CC$_{UCLUST}$ was 12.19 times faster than UCLUST. CC$_{SUMACLUST}$ outperforms SUMACLUST in all cases. The highest speed-up for CC$_{SUMACLUST}$ was observed to be 21.09x for the large metagenome dataset. The highest speedups for CC$_{SWARM}$ was also observed for the large metagenome dataset with 30 million reads. From these observations we conclude that Canopy clustering not only reduces the run time of an expensive clustering method but also scales well for larger benchmark datasets comparing to the smaller ones.   

\begin{figure}[t]	
	\begin{minipage}[t]{0.5\linewidth}
		\subfloat[]{
			\includegraphics[width=\linewidth]{bokulich_2.pdf}
		\label{fig:bokulich_2}}
	\end{minipage}%
	\hfill%
	\begin{minipage}[t]{0.5\linewidth}
		\subfloat[]{
			\includegraphics[width=\linewidth]{global_soil.pdf}
			\label{fig:global_soil}}
	\end{minipage}	
	\caption{Effect of number of processors on Runtime of CC$_{UCLUST}$, CC$_{SUMACLUST}$ and CC$_{SWARM}$ for Bokulich$_2$ and GLobal Soil datasets, two of the largest datasets used in this study.}
\end{figure}


\begin{figure}[t]	
	\begin{minipage}[t]{0.5\linewidth}
		\subfloat[]{
			\includegraphics[width=\linewidth]{bokulich_2_T1.pdf}
			\label{fig:bokulich_2_T1}}
	\end{minipage}%
	\hfill%
	\begin{minipage}[t]{0.5\linewidth}
		\subfloat[]{
			\includegraphics[width=\linewidth]{bokulich_2_T2.pdf}
			\label{fig:bokulich_2_T2}}
	\end{minipage}
	\caption{ \ref{fig:bokulich_2_T1} and \ref{fig:bokulich_2_T2} show effect of varying $T1$ and $T2$ on $F$-scores for the largest synthetic dataset Bokulich$_2$ }	
\end{figure}


\subsection{\textbf{Effect of Varying Number of Processors}}
Figure \ref{fig:bokulich_2}-\ref{fig:global_soil} shows the runtime of CC$_{UCLUST}$, CC$_{SUMACLUST}$ and CC$_{SWARM}$ on two largest benchmarks titled Bokulich$_2$ and Global Soil. We observe that increasing the number of processors reduces the total runtime. Steeper curves show reduction in runtime with increasing number of processors. Significant reductions in run time were observed CC$_{SUMACLUST}$ and CC$_{SUMACLUST}$ compared to CC$_{UCLUST}$.   

\subsection{\textbf{Sensitivity Analysis (Varying T1 and T2)}}
Figure \ref{fig:bokulich_2_T1}-\ref{fig:bokulich_2_T2} shows effect of varying $T1$ ($T2$ fixed at 0.34) and $T2$ ($T1$ fixed at 0.45) on F-scores for the largest synthetic dataset Bokulich$_2$. Reducing $T1$ leads to comparatively \textit{strict} soft-threshold which will reduce repetitions of instances in multiple canopies. For a fixed $T2$ this implies that lower $T1$ will yield better canopy assignment. From Figure \ref{fig:bokulich_2_T1} we can say that when $T1$'s range is in 0.4 to 0.6 our proposed approach provides better F-Scores. On the other hand, increasing $T2$ leads to comparatively \textit{relaxed} tight-threshold for canopies. As a result instances will be prematurely assigned to canopies without waiting for best match. From Figure \ref{fig:bokulich_2_T2} we can say that when $T2$'s range  is in 0.25 to 0.35 our proposed approach achieves better F-Scores. Lower $T2$ may lead to higher runtime since instances will continue to reappear until the $T2$ is met.                   


\section{Conclusion and Future Work}
\label{sec:Conclusion}

We developed a greedy approximate clustering process for any accurate and relatively expensive clustering on large scale metagenome datasets. Our approach takes advantage of the multi-core CPU systems by partitioning the large dataset with a fast and cheap pairwise distance measure and then deploying comparatively expensive clustering in parallel which considers only data points that are inside a partition. Our proposed approach scales well with large datasets and provide significant reduction in computation time. We demonstrate that our approach provides similar outcome in terms of biodiversity metrics, ground truth and taxonomic correlation with corresponding expensive clustering methods.


%\renewcommand{\bibfont}{\footnotesize}
\bibliographystyle{./IEEEtranBST/IEEEtran}
% argument is your BibTeX string definitions and bibliography database(s)
\bibliography{./IEEEtranBST/IEEEabrv,LSH-Canopy-Reference}

% that's all folks
\end{document}


